% LaTeX file for resume 
% This file uses the resume document class (res.cls)

\documentclass{res} 
%\usepackage{helvetica} % uses helvetica postscript font (download helvetica.sty)
%\usepackage{newcent}   % uses new century schoolbook postscript font 
\usepackage[margin=0.5in, footskip=0.25in]{geometry}
\newsectionwidth{0pt}  % So the text is not indented under section headings
%\usepackage{fancyhdr}  % use this package to get a 2 line header
%\renewcommand{\headrulewidth ,}{0pt} % suppress line drawn by default by fancyhdr
%\setlength{\headheight}{0pt} % allow room for 2-line header
%\setlength{\headsep}{24pt}  % space between header and text
%\setlength{\headheight}{24pt} % allow room for 2-line header
%\topmargin=-0.25in % start text higher on the page


\begin{document}
\thispagestyle{empty} % this page has no header  
\name{JEFFREY ROBERGE\\[0pt]}% the \\[12pt] adds a blank line after name

\address{ 101 Clarence Ave \\ Severna Park, MD 21146 }
\address{\textbf{E-mail} jeffrey.roberge@gmail.com \\ \textbf{Phone} (860) 754-6785}      
                                    

\begin{resume}
\vspace{-2pt}
\section{\centerline{OBJECTIVE}}
\vspace{2pt} % provide vertical space between section title and contents
Mechanical engineer with eight years of strong computational methods and software experience, looking to develop analytical applications for the purpose of aiding in structural and aero-thermal design of gas turbine engines.
 
%\vspace{0.2in}

\vspace{0pt} 
\section{\centerline{PROFESSIONAL EXPERIENCE}} 
\vspace{2pt}
{\sl Pratt \& Whitney}, East Hartford, CT \hfill        July 2024 - Present\\
Principal Mech. Eng., Systems Core Engineering Tools \& Methods (ED\&SI Core ETM)
  \vspace{4pt}
   \begin{itemize} \itemsep 1pt % reduce space between items
   \item Designed and developed pre- and post-processing tools for finite element analysis (FEA), with a focus on low and high cycle fatigue lifing systems, fracture mechanics, creep, strength and burst predictions.
   \item As mentor, developed new hire skills in Agile software, finite element analysis, and computational mechanics. 
   \item Partnered with Human Resources and the leadership team to source top engineering talent for engineering.
 \end{itemize} \vspace{-3pt}
 
{\sl Pratt \& Whitney}, East Hartford, CT \hfill        Dec 2020 - July 2024\\
Senior Mech. Eng., ED\&SI Core ETM
  \vspace{4pt}
   \begin{itemize} \itemsep 1pt % reduce space between items
   \item On a large multidisciplinary team, quantified powder nickel occlusion and fracture risk using a custom Monte Carlo simulation, and advised fleet management plans to mitigate this billion dollar flight safety issue.
   \item Increased engineers' analysis speed by developing nearly 40 software applications for a new FEA toolset. 
   \item Built training and up-skilled engineers, helping with technical issues in the new FEA solver environment.
 \end{itemize} \vspace{-3pt}

{\sl Pratt \& Whitney}, East Hartford, CT \hfill        Dec 2016 - Dec 2020\\
Mechanical Engineer, Engineering Development Program
\vspace{4pt}
 \begin{itemize} \itemsep 1pt
  \item  Analyzed military high pressure compressor (HPC) static structures and cases during a preliminary design phase
  \item  As part of the military HPC heat transfer group, iterated with the Design and Structures group to provide an HPC configuration that satisfies material capability and blade tip clearance requirements.
  \item Studied secondary flow structure by carrying out computational fluid dynamic analyses of the HPC's scavenge paths.
  \item Post-processed vibratory and thermal couple data from engine test to validate current modeling assumptions and limits.
  \item Performed part shape optimization via design of experiments and surrogate modeling.
 \end{itemize} \vspace{-3pt}
 
{\sl Structural Optimization Laboratory}, Storrs, CT (sol.engr@uconn.edu) \hfill                  Sept 2014 - Sept 2016\\
Graduate Assistant / Teaching Assistant, Department of Mechanical Engineering

  \begin{itemize} \itemsep 1pt
  \item Developed computational modeling algorithms for obtaining the effective properties of bone scaffold implants and composites, and for designing patient-specific bone scaffolds to expedite rehabilitation of critical size bone defects.
  \item Performed experimental studies to determine design for additive manufacturing rules for printed plastic components.
 \end{itemize}
\vspace{-3pt} 
\section{\centerline{ SKILLS }}
\vspace{2pt} 
\textbf{Programming:} Java, Python, C++, Perl, Lua, MatLab, Bash; Linux (RHEL, Ubuntu) and Windows\\ \textbf{Mechanical:} FEA, ANSYS MAPDL, NX, NXOpen API, Simcenter Multiphysics, Abaqus, ISight, SmartUQ; \\ \textbf{Software Skills:} Git, REST APIs, Agile Development and Scrum, Supercomputing, IBM LSF, LaTeX, Docker; \\ \textbf{Certifications:} AWS Cloud Practitioner
\vspace{-6pt} 
\section{\centerline{EDUCATION}} 
\vspace{2pt} 
{\sl Master of Science}, Mechanical Engineering \\
University of Connecticut, Storrs, CT \hspace{0.2in}  GPA 4.17/4.00 \hfill Sept 2016 \\
THESIS - Computational Design of Ceramic Bone Scaffolds Fabricated Via Direct Ink Writing
 
{\sl Bachelor of Science}, Biomedical Engineering \\ % \sl will be bold italic in
					 % New Century Schoolbook (or
					 % any postscript font) and
					 % just slanted in Computer
					 % Modern (default) font
University of Connecticut, Storrs, CT   \hspace{0.2in} GPA 3.76/4.00   \hfill    May 2013
%\\THESIS - Creation of a PID pressure controlled syringe pump to re-cellularize a rat lung extra cellular matrix
\vspace{-6pt} 
\section{\centerline{PUBLICATIONS}} 

\vspace{2pt}
	%\begin{itemize}
   %\item 
   Jeff Roberge and Juli\'{a}n Norato, "Computational design of curvilinear bone scaffolds fabricated via direct ink writing," Computer-Aided Design, Vol. 
    95, February 2018, Pages 1-13.
	%\end{itemize}

%\vspace{0.2in} 
 
%Joule Fellowship mentor, Department of Mechanical Engineering \hfill June 2015 - Aug 2015
%
%   \begin{itemize} \itemsep -2pt
%   \item Developed and led an NFS-funded research experience for teachers in topology optimization and 3D printing.
%   \end{itemize}

%\vspace{0in} 

%\section{\centerline{HONORS}} 
%\vspace{0pt}
%\begin{center}
%  Full Graduate tuition assistantship, University of Connecticut  \\
%                  Alpha Eta Mu Beta Biomedical Honors Society \\
%              Alpha Lambda Delta Honors Society 
%\end{center}
 
%\vspace{0.0in}
%\section{\centerline{INTERESTS}} 
%\vspace{0pt} 
%\begin{center}
%Piano, curling, mountain biking, rock climbing, weightlifting, running, cooking
%
%\end{center} 
 
\end{resume} 
\end{document}













